\documentclass[preprint, 3p,
authoryear]{elsarticle} %review=doublespace preprint=single 5p=2 column
%%% Begin My package additions %%%%%%%%%%%%%%%%%%%

\usepackage[hyphens]{url}

  \journal{Journal of Commodity Markets} % Sets Journal name

\usepackage{graphicx}
%%%%%%%%%%%%%%%% end my additions to header

\usepackage[T1]{fontenc}
\usepackage{lmodern}
\usepackage{amssymb,amsmath}
% TODO: Currently lineno needs to be loaded after amsmath because of conflict
% https://github.com/latex-lineno/lineno/issues/5
\usepackage{lineno} % add
\usepackage{ifxetex,ifluatex}
\usepackage{fixltx2e} % provides \textsubscript
% use upquote if available, for straight quotes in verbatim environments
\IfFileExists{upquote.sty}{\usepackage{upquote}}{}
\ifnum 0\ifxetex 1\fi\ifluatex 1\fi=0 % if pdftex
  \usepackage[utf8]{inputenc}
\else % if luatex or xelatex
  \usepackage{fontspec}
  \ifxetex
    \usepackage{xltxtra,xunicode}
  \fi
  \defaultfontfeatures{Mapping=tex-text,Scale=MatchLowercase}
  \newcommand{\euro}{€}
\fi
% use microtype if available
\IfFileExists{microtype.sty}{\usepackage{microtype}}{}
\usepackage[]{natbib}
\bibliographystyle{elsarticle-harv}

\ifxetex
  \usepackage[setpagesize=false, % page size defined by xetex
              unicode=false, % unicode breaks when used with xetex
              xetex]{hyperref}
\else
  \usepackage[unicode=true]{hyperref}
\fi
\hypersetup{breaklinks=true,
            bookmarks=true,
            pdfauthor={},
            pdftitle={European Carbon Market Connectedness and Risk Contagion: A Study of Return and Volatility Dynamics Between European Union Allowances (EUAs) and Financial Markets Between 2013 and 2025 and their Potential for Portfolio Diversification},
            colorlinks=false,
            urlcolor=blue,
            linkcolor=magenta,
            pdfborder={0 0 0}}

\setcounter{secnumdepth}{5}
% Pandoc toggle for numbering sections (defaults to be off)


% tightlist command for lists without linebreak
\providecommand{\tightlist}{%
  \setlength{\itemsep}{0pt}\setlength{\parskip}{0pt}}




\usepackage{subcaption, graphicx, pdflscape, float} \graphicspath{ {./figures/} }



\begin{document}


\begin{frontmatter}

  \title{European Carbon Market Connectedness and Risk Contagion: A
Study of Return and Volatility Dynamics Between European Union
Allowances (EUAs) and Financial Markets Between 2013 and 2025 and their
Potential for Portfolio Diversification}
    \author[CISL,Sigma]{Carlos Arcila Barrera%
  %
  }
   \ead{ca577@cantab.ac.uk} 
    \author[Caius]{Emre Usenmez%
  \corref{cor1}%
  }
   \ead{eu229@cam.ac.uk} 
      \affiliation[CISL]{
    organization={University of Cambridge Institute for Sustainability
Leadership},addressline={1 Regent Street},city={Cambridge},postcode={CB2
1GG},country={UK},}
    \affiliation[Sigma]{
    organization={Sigma Advanced Capital Management},addressline={203 N
La Salle Dr},city={Chicago},postcode={60601},state={IL},country={USA},}
    \affiliation[Caius]{
    organization={Gonville \& Caius College, University of
Cambridge},addressline={Trinity Street},city={Cambridge},postcode={CB2
1TA},country={UK},}
    \cortext[cor1]{Corresponding author}
  
  \begin{abstract}
  This paper uses Diebold-Yilmaz model to analyze the return and
  volatility connectedness between the European carbon market and the
  financial markets from the commencement of the 3rd phase of the EU
  emissions trading system in 2013 to the 4th phase until January 2025
  in order to ascertain the impact of fixed income, equity, commodities,
  and energy markets, as well as exogenous shocks and the recent reforms
  introduced under the Fit for 55 package and RePowerEU Plan. We examine
  the static and dynamic characteristics of the connectedness network
  and find that the return and volatility behavior of the European
  carbon market are primarily driven by their own fundamental factors.
  Thus it is largely independent of other financial markets except for
  coal and natural gas, and except during periods of financial stress
  where a relatively short-lived increase in the connectedness with
  other financial markets is observed. With such characteristics, EUAs
  can offer diversification benefits, especially for non-energy
  portfolios.
  \end{abstract}
    \begin{keyword}
    Carbon markets \sep emissions trading system \sep connectedness
measures \sep 
    risk diversification
  \end{keyword}
  
 \end{frontmatter}



\bibliography{EUAbibliography.bib}


\end{document}
